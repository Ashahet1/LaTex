\documentclass[11pt, letterpaper]{article}

\usepackage[margin=1in]{geometry}

\usepackage{amsmath,amsfonts,amssymb}


\def\eq1{y=\dfrac{x^3}{3x^3+x+1}}
\newcommand{\set}[1]{\setlength\itemsep{#1em}}

\begin{document}
    \textbf{Critical Thinking Questions}.
    \begin{enumerate}
        \set{1.2}
        \item Let's example the function in $\eq1$
        \item This is symbol for all real number $\mathbb{R}.$
        \item This is symbol for all rational number $\mathbb{Q}.$
        \item This is symbol for all integer number $\mathbb{Z}.$        
        \item Is it possible for a sequence to converge to two different numbers? If so, give an example. If not, why not?
        \item Explain hoinw to use partial sums to determine if a series converges or diverges. Give an example
        \item Explain why $\int\limits_{1}^{\infty} f(x)\,dx$ and $\sum\limits_{n=1}^{\infty} a_n$ need not converge to the same value, even if they are both convergent.
        \item In your own words Explain the alternating series remainder theorem. How is this theorem useful?
        \item Explain the difference between absolute and conditional convergence. Give an example of each?
        \item The Ratio test is inconclusive $ \displaystyle{\lim\limits_{n \to \infty} \left | \frac{a_{n+1}}{a_n}\right | =1}$. Give an example of one convergence series and one divergent series $\displaystyle{\lim\limits_{n \to \infty} \left | \frac{a_{n+1}}{a_n}\right | =1}$. Explain how you determined your examples.
        
    \end{enumerate}
    
\end{document}